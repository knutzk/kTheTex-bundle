% \iffalse meta-comment
%
% ktx-misc-style.dtx
% Copyright 2016 Knut Zoch <github.com/knutzk>
%
% This work may be distributed and/or modified under the conditions of
% the LaTeX Project Public License, either version 1.3 of this license
% or (at your option) any later version.  The latest version of this
% license is in http://www.latex-project.org/lppl.txt and version 1.3
% or later is part of all distributions of LaTeX version 2005/12/01 or
% later.
%
% This work has the LPPL maintenance status `maintained'.
%
% The Current Maintainer of this work is Knut Zoch.
%
% This work consists of the files kTheTex-bundle.dtx,
% kTheTex-bundle.ins, dtx/ktx-base.dtx, dtx/ktx-bibliography.dtx,
% dtx/ktx-debug.dtx, dtx/ktx-drafting.dtx, dtx/ktx-floats.dtx,
% dtx/ktx-font.dtx, dtx/ktx-headfoot.dtx, dtx/ktx-headings.dtx,
% dtx/ktx-misc-style.dtx, dtx/ktx-references.dtx, dtx/ktx-toc.dtx as
% well as the derived files ktxbbltx.sty, ktxreprt.cls and
% ktxthss.cls.
%
% \fi
%
% \iffalse
%<*driver>
\ProvidesFile{dtx/ktx-misc-style.dtx}
[2016/04/18 v0.1.0 ktx-misc-style]
%</driver>
%<*package>
\ProvidesPackage{ktxbbltx}
[2016/04/18 v0.1.0 kTheTex-bundle -- ktxbbltx package]
%</package>
%
%<*driver>
\documentclass[draft]{ltxdoc}
\EnableCrossrefs
\CodelineIndex
\RecordChanges
\changes{0.1.0}{2016/04/18}{Initial version} %
\GetFileInfo{dtx/ktx-misc-style.dtx} %
\DoNotIndex{} %
\title{The \textsf{kTheTex-bundle} file
  \textsf{ktx-misc-style.dtx}\thanks{This document corresponds to
    \textsf{ktx-misc-style.dtx}~\fileversion, dated \filedate.}}
\author{Knut Zoch \\ \texttt{github.com/knutzk}} %
\begin{document}
\maketitle
\DocInput{dtx/ktx-misc-style.dtx}
\end{document}
%</driver>
% \fi
%
% \CheckSum{0}
%
% \CharacterTable
%  {Upper-case    \A\B\C\D\E\F\G\H\I\J\K\L\M\N\O\P\Q\R\S\T\U\V\W\X\Y\Z
%   Lower-case    \a\b\c\d\e\f\g\h\i\j\k\l\m\n\o\p\q\r\s\t\u\v\w\x\y\z
%   Digits        \0\1\2\3\4\5\6\7\8\9
%   Exclamation   \!     Double quote  \"     Hash (number) \#
%   Dollar        \$     Percent       \%     Ampersand     \&
%   Acute accent  \'     Left paren    \(     Right paren   \)
%   Asterisk      \*     Plus          \+     Comma         \,
%   Minus         \-     Point         \.     Solidus       \/
%   Colon         \:     Semicolon     \;     Less than     \<
%   Equals        \=     Greater than  \>     Question mark \?
%   Commercial at \@     Left bracket  \[     Backslash     \\
%   Right bracket \]     Circumflex    \^     Underscore    \_
%   Grave accent  \`     Left brace    \{     Vertical bar  \|
%   Right brace   \}     Tilde         \~}
%
%
%
% \subsection{Miscellaneous Page Style}
% \label{sec:misc-page-style}
%
%   This section fixes the page layout completely. The base class was
%   called with DIV=9, i.e. the available space (horizontally and
%   vertically) is divided in 9x9 equal sections. Six out of nine
%   sections are used for text (inner margin: 1, outer margin: 2, top
%   margin: 1, bottom margin: 2).
%
%   For a font size of 11pt, use a leading of 14pt (requested with the
%   package |leading|). Then, we want the text to be longer than
%   usually. Consider the following calculation: One A4 page measures
%   a width of 210mm, i.e. the text is spread over 140mm. Now multiply
%   this value with the golden section for a nice text height. The
%   height then is 226.5mm or 644.5pt (1mm = 2.845pt). Subtracting
%   about 30pt for the header which should be included in these
%   calculations, there are 614.5pt left for the main text. With a
%   leading of 14pt, 44 lines therefore is a good length. This yields
%   a text height of 613pt which is close to the ideal value (616
%   minus 3 because the last line does not need leading).
%
%   The top margin is modified as well: Set it to be the same as the
%   inner margin. This corresponds to roughly 59.5pt (1in is the Latex
%   standard offset).
%
%   Skips between paragraphs are set to 0pt (with no variable space).
%   A more relaxed option would be:
%   |\setlength{\parskip}{0pt plus 0.3\baselineskip}|.
%
%    \begin{macrocode}
\RequirePackage{leading}
\leading{14pt}
\setlength{\topmargin}{59.5pt-1in}
\setlength{\textheight}{613pt}
\setlength{\parskip}{0pt}
%    \end{macrocode}
%
% \begin{macro}{format}
%   The calculation of the text width and height as described above
%   follows recommendations from typography and guarantees an optimal
%   text spread. One of the most important aspects in this case is the
%   readibility of lines, i.e. the line length should (on average) not
%   exceed a certain number of characters.
%
%   Besides, the layout uses both the very classical division of the
%   page into 9x9 fields as well as the golden section for text width
%   and height. However, this class also provides modifications of
%   these settings with the |format| option.
%
%   The option |format=enlarge| increases the text spread over the
%   page while switching to a font size of 12pt (in the standard
%   setting the size is 11pt). To maintain optimal readability, the
%   page is divided to 10x10 fields instead, 7 of which are used for
%   the main text. Again, the text height is calculated following the
%   golden section. The leading is extended to 16.15pt which allows 40
%   lines of text in the given text height of 642pt.
%    \begin{macrocode}
\IfEqCase*{\ktx@format}{%
  {standard}{\relax}%
  {enlarge}{%
    \KOMAoptions{fontsize=12pt, DIV=10}
    \leading{16.15pt}
    \setlength{\topmargin}{49pt-1in}
    \setlength{\textheight}{642pt}}%
}[\ktx@error{Unknown format: "\ktx@format"}{}]
%    \end{macrocode}
% \end{macro}
%
%
% \begin{macro}{xcolor}
%   This section defines three different colours by using the package
%   |xcolor|. The colours are used for link marking within hyperref
%   (as long as the package is loaded). They can also be used for
%   figures etc. to stick to a consistent colour scheme. The used
%   colours are names ``Venetian'', ``Lime'' and ``Dodger''. In the
%   end, redefine the title command to be coloured.
%
%   Other colours, currently not used anymore:
%
%   \begin{itemize}
%   \item |\definecolor{Maroon}{cmyk}{0, 0.87, 0.68, 0.32}|
%   \item |\definecolor{RoyalBlue}{cmyk}{1, 0.5, 0, 0}|
%   \item |\definecolor{Black}{cmyk}{0, 0, 0, 0}|
%   \item |\definecolor{webgreen}{cmyk}{1, 0, 1, 0.5}|
%   \item |\definecolor{webbrown}{cmyk}{0, 1, 1, 0.4}|
%   \end{itemize}
%
%    \begin{macrocode}
\RequirePackage{xcolor}
\definecolor{Venetian}{cmyk}{0, 0.95, 0.85, 0.30}
\definecolor{Lime}{cmyk}{0.85, 0, 0.75, 0.25}
\definecolor{Dodger}{cmyk}{1, 0.40, 0, 0.10}
%    \end{macrocode}
% \end{macro}
%
% \begin{macro}{microtype}
%   Optimise text spread with microtype package. First make a check
%   whether the |microtype| option has been set. Then load package
%   |microtype|. Options include: use language information from the
%   babel package to determine kerning etc. Set stretch and shrink to
%   a value of 10 (20 is default). And always use final=true,
%   i.e. ignore draft options from the class in order to \emph{always}
%   run |microtype|. To really turn it off, use the option
%   |microtype=false|.
%
%    \begin{macrocode}
\ifthenelse{\boolean{ktx@microtype}}{%
  \RequirePackage[babel   = true,%
                  stretch = 10,%
                  shrink  = 10,%
                  final   = true,%
                  ]{microtype}
 }{%
   \ktx@warning{Microtype package not loaded.}
}
%    \end{macrocode}
% \end{macro}
%
% \begin{macro}{letterspacing}
%   Now adjust the letter spacing for the small-caps font (in the
%   standard settings, the spacing is way too small). For that, use
%   the |microtype| package. First, check if |microtype| was loaded or
%   not. If this is not the case, load the package |letterspacing|
%   which is part of the |microtype| package. Maybe the latter one was
%   turned off deliberately, so we only want the letter spacing
%   features here then.
%
%   Provide commands for fully-capitalised and small-caps-only
%   fonts. By default, those commands work fine, but always issue a
%   warning when the spacing is not adjusted. If either |letterspace|
%   or |microtype| were loaded, apply the spacing adjustments.
%
%   Finally, provide global commands |\textsca| and |\textscl| to use
%   all-caps and small-caps.
%
%    \begin{macrocode}
\@ifpackageloaded{microtype}{%
  \relax
}{%
  \RequirePackage{letterspace}}
\newrobustcmd{\ktx@allcaps}[1]{%
  \ktx@warning{Spacing of small caps not adjusted! Using
    default\MessageBreak spacing now.}\textsc{\MakeUppercase{#1}}}
\newrobustcmd{\ktx@lowcaps}[1]{%
  \ktx@warning{Spacing of small caps not adjusted! Using
    default\MessageBreak spacing now.}\textsc{\MakeLowercase{#1}}}
\ifboolexpr{ test {\@ifpackageloaded{letterspace}} or
             test {\@ifpackageloaded{microtype}} }{%
  \ktx@info{Using letterspacing options of microtype package.}
  \renewrobustcmd{\ktx@allcaps}[1]{%
                  \textls[150]{\MakeUppercase{#1}}}
  \renewrobustcmd{\ktx@lowcaps}[1]{%
                  \textls[50]{\textsc{\MakeLowercase{#1}}}}
  \let\ktx@textsc@old\textsc
  \renewcommand*{\textsc}[1]{\textls[50]{\ktx@textsc@old{#1}}}
}{%
  \relax
}
\let\textsca\ktx@allcaps
\let\textscl\ktx@lowcaps
%    \end{macrocode}
% \end{macro}
%
%
% \begin{macro}{swash font}
%   Some fonts provide a specific typeface, called \emph{swashed}
%   (similar to italics). These commands define |textsw| and |swshape|
%   for the class and -- when the fonts MinionPro and MyriadPro are
%   used -- applies their definition of them. Otherwise, those two
%   commands equal |textit| and |itshape|.
%
%    \begin{macrocode}
\newcommand*{\ktx@textsw}[1]{\textit{#1}}
\newcommand*{\ktx@swshape}{\itshape}
\ifdefstring{\ktx@font}{minion}{%
  \renewcommand*{\ktx@textsw}[1]{\textsw{#1}}
  \renewcommand*{\ktx@swshape}{\swshape}
}{}
%    \end{macrocode}
% \end{macro}
%
%
% \begin{macro}{Other packages}
%   Various packages applying the following settings:
%
%   \begin{itemize}
%   \item encoding of document is UTF8
%   \item language support via babel (hyphenation etc.)
%   \item math package of the AMS
%   \item sync quoting style with language by using |\enquote{...}|
%   \item support for graphics inclusion
%   \item custom commands do not eat spaces
%   \end{itemize}
%    \begin{macrocode}
\RequirePackage[utf8]{inputenc}
\RequirePackage{babel}
\RequirePackage{amsmath}
\RequirePackage[autostyle=true]{csquotes}
\RequirePackage{graphicx}
\RequirePackage{xspace}
%    \end{macrocode}
% \end{macro}
%
%
% \begin{macro}{enumitem}
%   The package |enumitem| controls the behaviour of lists. 
%    \begin{macrocode}
\RequirePackage{enumitem}
\setlist{nosep}
\setlist{leftmargin=1.5em}
%    \end{macrocode}
% \end{macro}
%
%
% \begin{macro}{numbering} 
%   Control the numbering of equations, tables and figures. The number
%   of digits assigned to those objects is set to depend on the used
%   class: theses use chapter-wide numbers, report use section-wide
%   numbers.
%
%   The following might be an interesting addition: change the
%   numbering of sections (+subsections etc.) to numbers without
%   chapter, so chapters can be omitted
%   completely. |\renewcommand*\thesection{\arabic{section}}|
%    \begin{macrocode}
%<*report>
\numberwithin{equation}{section}
\numberwithin{table}{section}
\numberwithin{figure}{section}
%</report>
%<*thesis>
\numberwithin{equation}{chapter}
\numberwithin{table}{chapter}
\numberwithin{figure}{chapter}
%</thesis>
%    \end{macrocode}
% \end{macro}
%
%
%
% \subsection{Unused Packages}
% \label{sec:unused-packages}
%
% There are a number of packages that initially were part of the
% template, but are not of any use right now. Those include:
% \begin{verbatim}
%  Currently not needed packages (delete them?)
%  \RequirePackage{comment}                % comment whole parts of text
%  \RequirePackage[shortcuts]{extdash}     % dash commands for compound words
%  \RequirePackage{float}                  % force floats to put "here" [!H]
%  \RequirePackage{multirow}               % multi-row support for tables
%  \RequirePackage{rotating}               % flip floats
%  \RequirePackage{todonotes}              % add to-do notes to text
%  \RequirePackage{units}                  % correct units typesetting
%  \RequirePackage{wrapfig}                % wrap text around figures
%  \RequirePackage{datetime}
% \end{verbatim}
%
% The very last package allows customisation of dates:
% \begin{itemize}
% \item Fix dates with: |\newdate{name}{dd}{mm}{yyyy}|
% \item Display dates with: |\displaydate{name} (or the usual \today)|
% \item Customise appearance e.g. with |\ddmmyyyydate| |\longdate|
% \item Change separator (e.g. ".", "/") via |\renewcommand{\dateseparator}{xxx}|
% \end{itemize}
%
%
%
% \subsection{Abstracts}
% \label{sec:abstract}
%
% \begin{macro}{abstract}
% Depending on the type of document, there should also be an abstract
% just after the title. The following code introduces a dedicated
% environment that can be used to set abstracts.
%    \begin{macrocode}
%<*thesis>
\newenvironment{abstractpage}[1][1.0]{%
  \begin{samepage}%
  \thispagestyle{plain}%
  \vspace*{1em minus 1em}%
  \begin{center}\begin{minipage}{#1\linewidth}}{%
  \end{minipage}\end{center}%
  \vspace*{0pt minus 2em}%
  \end{samepage}
  \cleardoublepage}
\newenvironment{abstract}[2]{%
  \addsec*{#2}\begin{otherlanguage}{#1}}{%
  \end{otherlanguage}%
  \vspace*{2em}%
}
%</thesis>
%    \end{macrocode}
% \end{macro}
%
%
%
%
% \subsection{Declaration}
% \label{sec:declaration}
%
% At the end of a thesis, there should also be a declaration. The
% following code allows to set such a declaration at the end of the
% thesis document.
%    \begin{macrocode}
%<*thesis>
\newcommand{\ktxDeclaration}[2]{%
\newdate{enddate}{\ktx@tp@enddate@dd}{\ktx@tp@enddate@mm}{\ktx@tp@enddate@yyyy}
  \AtEndDocument{%
    \begin{otherlanguage}{#1}
    \clearpage\thispagestyle{empty}
    \null\vfill\noindent
    \begin{minipage}[t]{0.225\textwidth}
      \hspace*{1em}\bfseries\large%
      \IfLanguageName{ngerman}{%
        Erkl{\"a}rung%
      }{%
        Declaration%
      }
    \end{minipage}%
    \begin{minipage}[t]{0.775\textwidth}%
      \IfLanguageName{ngerman}{%
        nach #2:\\[1em]
        Hiermit erkl{\"a}re ich, dass ich diese Abschlussarbeit selbst{\"a}ndig
        verfasst habe, keine anderen als die angegebenen Quellen und Hilfsmittel
        benutzt habe und alle Stellen, die w{\"o}rtlich oder sinngem{\"a}\ss{}
        aus ver{\"o}ffentlichten Schriften entnommen wurden, als solche
        kenntlich gemacht habe.\par
        Dar{\"u}berhinaus erkl{\"a}re ich, dass diese Abschlussarbeit nicht,
        auch nicht auszugsweise, im Rahmen einer nichtbestandenen Pr{\"u}fung
        an dieser oder einer anderen Hochschule eingereicht wurde.\\[1em]
        \begin{center}\ktx@tp@instaddress, den \displaydate{enddate}\end{center}
      }{%
        according to #2:\\[1em]
        This dissertation is the result of my own work and includes nothing
        which is the outcome of work done in collaboration except as declared in
        the Preface and specified in the text.\par
        It is not substantially the same as any that I have submitted, or, is
        being concurrently submitted for a degree or diploma or other
        qualification at this or any other University or similar institution
        except as declared in the Preface and specified in the text. I further
        state that no substantial part of my dissertation has already been
        submitted, or, is being concurrently submitted for any such degree,
        diploma or other qualification at this or any other University or
        similar institution except as declared in the Preface and specified in
        the text.\\[1em]
        \begin{center}\ktx@tp@instaddress, \displaydate{enddate}\end{center}
      }%
      \vspace*{1.5cm}
      \begin{center}(\ktx@tp@author)\end{center}%
      \vspace*{1\baselineskip}%
    \end{minipage}%
    \end{otherlanguage}
}}
%</thesis>
%    \end{macrocode}
%