% \iffalse meta-comment
%
% ktx-misc-style.dtx
% Copyright 2016 K. Zoch <github.com/kzoch>
%
% This work may be distributed and/or modified under the conditions of
% the LaTeX Project Public License, either version 1.3 of this license
% or (at your option) any later version.  The latest version of this
% license is in http://www.latex-project.org/lppl.txt and version 1.3
% or later is part of all distributions of LaTeX version 2005/12/01 or
% later.
%
% This work has the LPPL maintenance status `maintained'.
%
% The Current Maintainer of this work is K. Zoch.
%
% This work consists of the files kTheTex-bundle.dtx,
% kTheTex-bundle.ins, dtx/ktx-base.dtx, dtx/ktx-bibliography.dtx,
% dtx/ktx-debug.dtx, dtx/ktx-drafting.dtx, dtx/ktx-floats.dtx,
% dtx/ktx-font.dtx, dtx/ktx-headfoot.dtx, dtx/ktx-headings.dtx,
% dtx/ktx-misc-style.dtx, dtx/ktx-references.dtx, dtx/ktx-toc.dtx as
% well as the derived files ktxbbltx.sty, ktxreprt.cls and
% ktxthss.cls.
%
% \fi
%
% \iffalse
%<*driver>
\ProvidesFile{dtx/ktx-misc-style.dtx}
[2016/04/18 v0.1.0 ktx-misc-style]
%</driver>
%<*package>
\ProvidesPackage{ktxbbltx}
[2016/04/18 v0.1.0 kTheTex-bundle -- ktxbbltx package]
%</package>
%
%<*driver>
\documentclass[draft]{ltxdoc}
\EnableCrossrefs
\CodelineIndex
\RecordChanges
\changes{0.1.0}{2016/04/18}{Initial version} %
\GetFileInfo{dtx/ktx-misc-style.dtx} %
\DoNotIndex{} %
\title{The \textsf{kTheTex-bundle} file
  \textsf{ktx-misc-style.dtx}\thanks{This document corresponds to
    \textsf{ktx-misc-style.dtx}~\fileversion, dated \filedate.}}
\author{K. Zoch \\ \texttt{github.com/kzoch}} %
\begin{document}
\maketitle
\DocInput{dtx/ktx-misc-style.dtx}
\end{document}
%</driver>
% \fi
%
% \CheckSum{0}
%
% \CharacterTable
%  {Upper-case    \A\B\C\D\E\F\G\H\I\J\K\L\M\N\O\P\Q\R\S\T\U\V\W\X\Y\Z
%   Lower-case    \a\b\c\d\e\f\g\h\i\j\k\l\m\n\o\p\q\r\s\t\u\v\w\x\y\z
%   Digits        \0\1\2\3\4\5\6\7\8\9
%   Exclamation   \!     Double quote  \"     Hash (number) \#
%   Dollar        \$     Percent       \%     Ampersand     \&
%   Acute accent  \'     Left paren    \(     Right paren   \)
%   Asterisk      \*     Plus          \+     Comma         \,
%   Minus         \-     Point         \.     Solidus       \/
%   Colon         \:     Semicolon     \;     Less than     \<
%   Equals        \=     Greater than  \>     Question mark \?
%   Commercial at \@     Left bracket  \[     Backslash     \\
%   Right bracket \]     Circumflex    \^     Underscore    \_
%   Grave accent  \`     Left brace    \{     Vertical bar  \|
%   Right brace   \}     Tilde         \~}
%
%
%
% \subsection{Miscellaneous Page Style}
% \label{sec:misc-page-style}
%
% \begin{macro}{pagelayout}
%   This section fixes the page layout completely. The base class was
%   called with DIV=9, i.e. the available space (horizontally and
%   vertically) is divided in 9x9 equal sections. Six out of nine
%   sections are used for text (inner margin: 1, outer margin: 2, top
%   margin: 1, bottom margin: 2).
%
%   For a font size of 11pt, use a leading of 14pt (requested with the
%   package |leading|). Then, we want the text to be longer than
%   usually. Consider the following calculation: One A4 page measures
%   a width of 210mm, i.e. the text is spread over 140mm. Now multiply
%   this value with the golden section for a nice text height. The
%   height then is 226.5mm or 595pt (1mm = 2.845pt). With a leading of
%   14pt, this equals 42 lines (-2pt because last line does not need
%   leading).
%
%   The top margin is modified as well: Set it to be the same as the
%   inner margin. This corresponds to 1/6 of the text width (1in is
%   the Latex standard offset).
%
%   Skips between paragraphs are set to 0pt (with no variable space).
%   A more relaxed option would be:
%   |\setlength{\parskip}{0pt plus 0.3\baselineskip}|.
%
%    \begin{macrocode}
\RequirePackage{leading}
\leading{14pt}
\setlength{\textheight}{42\baselineskip-2pt}
\setlength{\topmargin}{0.16666\textwidth-1in}
\setlength{\parskip}{0pt}
%    \end{macrocode}
% \end{macro}
%
%
% \begin{macro}{xcolor}
%   This section defines three different colours by using the package
%   |xcolor|. The colours are used for link marking within hyperref
%   (as long as the package is loaded). They can also be used for
%   figures etc. to stick to a consistent colour scheme. The used
%   colours are names ``Venetian'', ``Lime'' and ``Dodger''. In the
%   end, redefine the title command to be coloured.
%
%   Other colours, currently not used anymore:
%
%   \begin{itemize}
%   \item |\definecolor{Maroon}{cmyk}{0, 0.87, 0.68, 0.32}|
%   \item |\definecolor{RoyalBlue}{cmyk}{1, 0.5, 0, 0}|
%   \item |\definecolor{Black}{cmyk}{0, 0, 0, 0}|
%   \item |\definecolor{webgreen}{cmyk}{1, 0, 1, 0.5}|
%   \item |\definecolor{webbrown}{cmyk}{0, 1, 1, 0.4}|
%   \end{itemize}
%
%    \begin{macrocode}
\RequirePackage{xcolor}
\definecolor{Venetian}{cmyk}{0, 0.95, 0.85, 0.30}
\definecolor{Lime}{cmyk}{0.85, 0, 0.75, 0.25}
\definecolor{Dodger}{cmyk}{1, 0.40, 0, 0.10}
\let\oldtitle\title
\renewcommand*{\title}[1]{%
  \oldtitle{\color{Venetian}#1}
}
%    \end{macrocode}
% \end{macro}
%
% \begin{macro}{microtype}
%   Optimise text spread with microtype package. First make a check
%   whether the |microtype| option has been set. Then load package
%   |microtype|. Options include: use language information from the
%   babel package to determine kerning etc. Set stretch and shrink to
%   a value of 10 (20 is default). And always use final=true,
%   i.e. ignore draft options from the class in order to \emph{always}
%   run |microtype|. To really turn it off, use the option
%   |microtype=false|.
%
%    \begin{macrocode}
\ifthenelse{\boolean{ktx@microtype}}{%
  \RequirePackage[babel   = true,%
                  stretch = 10,%
                  shrink  = 10,%
                  final   = true,%
                  ]{microtype}
 }{%
   \ktx@warning{Microtype package not loaded.}
}
%    \end{macrocode}
% \end{macro}
%
% \begin{macro}{letterspacing}
%   Now adjust the letter spacing for the small-caps font (in the
%   standard settings, the spacing is way too small). For that, use
%   the |microtype| package. First, check if |microtype| was loaded or
%   not. If this is not the case, load the package |letterspacing|
%   which is part of the |microtype| package. Maybe the latter one was
%   turned off deliberately, so we only want the letter spacing
%   features here then.
%
%   Provide commands for fully-capitalised and small-caps-only
%   fonts. By default, those commands work fine, but always issue a
%   warning when the spacing is not adjusted. If either |letterspace|
%   or |microtype| were loaded, apply the spacing adjustments.
%
%   Finally, provide global commands |\textsca| and |\textscl| to use
%   all-caps and small-caps.
%
%    \begin{macrocode}
\@ifpackageloaded{microtype}{
  \relax
}{%
  \RequirePackage{letterspace}}
\newrobustcmd{\ktx@allcaps}[1]{%
  \ktx@warning{Spacing of small caps not adjusted! Using
    default\MessageBreak spacing now.}\textsc{\MakeUppercase{#1}}}
\newrobustcmd{\ktx@lowcaps}[1]{%
  \ktx@warning{Spacing of small caps not adjusted! Using
    default\MessageBreak spacing now.}\textsc{\MakeLowercase{#1}}}
\ifboolexpr{ test {\@ifpackageloaded{letterspace}} or
             test {\@ifpackageloaded{microtype}} }{%
  \ktx@info{Using letterspacing options of microtype package.}
  \renewrobustcmd{\ktx@allcaps}[1]{%
                  \textls[150]{\MakeUppercase{#1}}}
  \renewrobustcmd{\ktx@lowcaps}[1]{%
                  \textls[50]{\textsc{\MakeLowercase{#1}}}}
  \let\ktx@textsc@old\textsc
  \renewcommand*{\textsc}[1]{\textls[50]{\ktx@textsc@old{#1}}}
}{%
  \relax
}
\let\textsca\ktx@allcaps
\let\textscl\ktx@lowcaps
%    \end{macrocode}
% \end{macro}
%
%
% \begin{macro}{swash font}
%   Some fonts provide a specific typeface, called \emph{swashed}
%   (similar to italics). These commands define |textsw| and |swshape|
%   for the class and -- when the fonts MinionPro and MyriadPro are
%   used -- applies their definition of them. Otherwise, those two
%   commands equal |textit| and |itshape|.
%
%    \begin{macrocode}
\newcommand*{\ktx@textsw}[1]{\textit{#1}}
\newcommand*{\ktx@swshape}{\itshape}
\ifdefstring{\ktx@font}{minion}{%
  \renewcommand*{\ktx@textsw}[1]{\textsw{#1}}
  \renewcommand*{\ktx@swshape}{\swshape}
}{}
%    \end{macrocode}
% \end{macro}
%
%
% \begin{macro}{Other packages}
%   Various packages applying the following settings:
%
%   \begin{itemize}
%   \item encoding of document is UTF8
%   \item language support via babel (hyphenation etc.)
%   \item math package of the AMS
%   \item sync quoting style with language by using |\enquote{...}|
%   \item support for graphics inclusion
%   \item custom commands do not eat spaces
%   \end{itemize}
%    \begin{macrocode}
\RequirePackage[utf8]{inputenc}
\RequirePackage{babel}
\RequirePackage{amsmath}
\RequirePackage[autostyle=true]{csquotes}
\RequirePackage{graphicx}
\RequirePackage{xspace}
%    \end{macrocode}
% \end{macro}
%
%
% \begin{macro}{enumitem}
%   The package |enumitem| controls the behaviour of lists. 
%    \begin{macrocode}
\RequirePackage{enumitem}
\setlist{nosep}
\setlist{leftmargin=1.5em}
%    \end{macrocode}
% \end{macro}
%
%
% \begin{macro}{numbering} 
%   Control the numbering of equations, tables and figures. The number
%   of digits assigned to those objects is set to depend on the used
%   class: theses use chapter-wide numbers, report use section-wide
%   numbers.
%
%   The following might be an interesting addition: change the
%   numbering of sections (+subsections etc.) to numbers without
%   chapter, so chapters can be omitted
%   completely. |\renewcommand*\thesection{\arabic{section}}|
%    \begin{macrocode}
%<*report>
\numberwithin{equation}{section}
\numberwithin{table}{section}
\numberwithin{figure}{section}
%</report>
%<*thesis>
\numberwithin{equation}{chapter}
\numberwithin{table}{chapter}
\numberwithin{figure}{chapter}
%</thesis>
%    \end{macrocode}
% \end{macro}
%
%
%
% \subsection{Unused Packages}
% \label{sec:unused-packages}
%
% There are a number of packages that initially were part of the
% template, but are not of any use right now. Those include:
% \begin{verbatim}
%  Currently not needed packages (delete them?)
%  \RequirePackage{comment}                % comment whole parts of text
%  \RequirePackage[shortcuts]{extdash}     % dash commands for compound words
%  \RequirePackage{float}                  % force floats to put "here" [!H]
%  \RequirePackage{multirow}               % multi-row support for tables
%  \RequirePackage{rotating}               % flip floats
%  \RequirePackage{todonotes}              % add to-do notes to text
%  \RequirePackage{units}                  % correct units typesetting
%  \RequirePackage{wrapfig}                % wrap text around figures
%  \RequirePackage{datetime}
% \end{verbatim}
%
% The very last package allows customisation of dates:
% \begin{itemize}
% \item Fix dates with: |\newdate{name}{dd}{mm}{yyyy}|
% \item Display dates with: |\displaydate{name} (or the usual \today)|
% \item Customise appearance e.g. with |\ddmmyyyydate| |\longdate|
% \item Change separator (e.g. ".", "/") via |\renewcommand{\dateseparator}{xxx}|
% \end{itemize}
%

