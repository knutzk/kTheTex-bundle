% \iffalse meta-comment
%
% ktx-references.dtx
% Copyright 2016 K. Zoch <github.com/knutzk>
%
% This work may be distributed and/or modified under the conditions of
% the LaTeX Project Public License, either version 1.3 of this license
% or (at your option) any later version.  The latest version of this
% license is in http://www.latex-project.org/lppl.txt and version 1.3
% or later is part of all distributions of LaTeX version 2005/12/01 or
% later.
%
% This work has the LPPL maintenance status `maintained'.
%
% The Current Maintainer of this work is K. Zoch.
%
% This work consists of the files kTheTex-bundle.dtx,
% kTheTex-bundle.ins, dtx/ktx-base.dtx, dtx/ktx-bibliography.dtx,
% dtx/ktx-debug.dtx, dtx/ktx-drafting.dtx, dtx/ktx-floats.dtx,
% dtx/ktx-font.dtx, dtx/ktx-headfoot.dtx, dtx/ktx-headings.dtx,
% dtx/ktx-misc-style.dtx, dtx/ktx-references.dtx, dtx/ktx-toc.dtx as
% well as the derived files ktxbbltx.sty, ktxreprt.cls and
% ktxthss.cls.
%
% \fi
%
% \iffalse
%<*driver>
\ProvidesFile{dtx/ktx-references.dtx}
[2016/04/18 v0.1.0 ktx-references]
%</driver>
%
%<*driver>
\documentclass[draft]{ltxdoc}
\EnableCrossrefs
\CodelineIndex
\RecordChanges
\changes{v0.1.0}{2016/04/18}{Initial version} %
\GetFileInfo{dtx/ktx-references.dtx} %
\DoNotIndex{} %
\title{The \textsf{kTheTex-bundle} file
  \textsf{ktx-references.dtx}\thanks{This document corresponds to
    \textsf{ktx-references.dtx}~\fileversion, dated \filedate.}}
\author{K. Zoch \\ \texttt{github.com/knutzk}} %
\begin{document}
\maketitle
\DocInput{dtx/ktx-references.dtx}
\end{document}
%</driver>
% \fi
%
% \CheckSum{0}
%
% \CharacterTable
%  {Upper-case    \A\B\C\D\E\F\G\H\I\J\K\L\M\N\O\P\Q\R\S\T\U\V\W\X\Y\Z
%   Lower-case    \a\b\c\d\e\f\g\h\i\j\k\l\m\n\o\p\q\r\s\t\u\v\w\x\y\z
%   Digits        \0\1\2\3\4\5\6\7\8\9
%   Exclamation   \!     Double quote  \"     Hash (number) \#
%   Dollar        \$     Percent       \%     Ampersand     \&
%   Acute accent  \'     Left paren    \(     Right paren   \)
%   Asterisk      \*     Plus          \+     Comma         \,
%   Minus         \-     Point         \.     Solidus       \/
%   Colon         \:     Semicolon     \;     Less than     \<
%   Equals        \=     Greater than  \>     Question mark \?
%   Commercial at \@     Left bracket  \[     Backslash     \\
%   Right bracket \]     Circumflex    \^     Underscore    \_
%   Grave accent  \`     Left brace    \{     Vertical bar  \|
%   Right brace   \}     Tilde         \~}
%
%
%
% \subsection{References}
% \label{sec:impl-refer}
%
% The following code will include the |hyperref| package, which should
% always be the last package loaded, as well as the package
% |cleveref|.
%
%
% \begin{macro}{option hyperref}
%   The package will only be loaded when the corresponding option is
%   set to |hyperref=true|. In case the user wants to add a package
%   that conflicts with hyperref if its loaded first, turn hyperref
%   off via |hyperref=false|. The following code will define some
%   colours to mark links within the document in non-standard
%   colours. Afterwards, |hyperref| is loaded with the following
%   options: 
%
% \begin{itemize}
% \item |colorlinks|: use coloured links instead of coloured boxes
%   around the links.
% \item |urlcolor|, |citecolor|, |linkcolor| set the corresponding
%   colours to the ones just defined.
% \item Link the page numbers in the TOC, instead of the chapters
%   themselves. 
% \end{itemize}
%
% If option |web| is given to the class, do everything as described
% above. For option "print", hide links, i.e. make them black again.
%
%    \begin{macrocode}
\ifthenelse{\boolean{ktx@hyperref}}{%
 \RequirePackage[%
    colorlinks = true,%
    urlcolor   = Venetian,%
    citecolor  = Lime,%
    linkcolor  = Dodger,%
    linktocpage = true,%
    ]{hyperref}
  \ifthenelse{\boolean{ktx@web}}{%
    \relax
  }{
    \hypersetup{hidelinks}
  }  
%    \end{macrocode}
% \end{macro}
%
% \begin{macro}{cleveref}
%   Now also call the cleveref package for intelligent
%   references. They can be used via |\cref{}| commands instead of
%   |\ref{}| and include the \emph{type} of the reference within the
%   reference itself (also linked). This is achieved with the option
%   |nameinlink|. Option |capitalise| used capitalised version for
%   every type of reference, e.g. Eq. (3), Section 4, Table 3.
%
%   For some reason, some of the German reference names are not set
%   correctly, therefore some adjustments are being made.
%
%    \begin{macrocode}
  \RequirePackage[nameinlink, capitalise]{cleveref}
  \addto\captionsngerman{
    \if@cref@abbrev
    \crefname{equation}{Gl.}{Gl.}
    \Crefname{equation}{Gleichung}{Gleichungen}
    \crefname{table}{Tab.}{Tab.}
    \Crefname{table}{Tabelle}{Tabellen}
    \fi
  }
}{%
  \relax
}
%    \end{macrocode}
% \end{macro}
%

