% \subsection{References}
% \label{sec:impl-refer}
%
% The following code will include the |hyperref| package, which should always be
% the last package loaded, as well as the package |cleveref|.
% \begin{macro}{option hyperref}
%   The package will only be loaded when the corresponding option is set to
%   |hyperref=true|. In case the user wants to add a package that conflicts with
%   hyperref if its loaded first, turn hyperref off via |hyperref=false|.
%    \begin{macrocode}
\ifthenelse{\boolean{ktx@hyperref}}{%
%    \end{macrocode}
% Defining some colours now to mark the links within the document in
% non-standard colours.
%    \begin{macrocode}
  \definecolor{Maroon}{cmyk}{0, 0.87, 0.68, 0.32}
  \definecolor{RoyalBlue}{cmyk}{1, 0.50, 0, 0}
  \definecolor{Black}{cmyk}{0, 0, 0, 0}
  \definecolor{webgreen}{rgb}{0,.5,0}
  \definecolor{webbrown}{rgb}{.6,0,0}
%    \end{macrocode}
% The following options are called: Do not use colour boxes for linking, but
% coloured text instead. Use the specified colours for URL, citations and
% links. Additionally, do not link the names in the TOC, but rather the page
% numbers.
%    \begin{macrocode}
  \RequirePackage[%
          colorlinks = true,%
          urlcolor   = webbrown,%
          citecolor  = webgreen,%
          linkcolor  = RoyalBlue,%
          linktocpage = true,%
          ]{hyperref}
%    \end{macrocode}
% Now also call the cleveref package for intelligent references. They can be
% used via |\cref{}| commands instead of |\ref{}| and include the \emph{type} of
% the reference within the reference itself (also linked).
%    \begin{macrocode}
  \RequirePackage[nameinlink, capitalise]{cleveref}
%    \end{macrocode}
% Apparently, some of the German abbreviations for equations and tables were
% missing. This is a temporary hack to add them to |cleveref|.
%    \begin{macrocode}
  \addto\captionsngerman{
    \if@cref@abbrev
    \crefname{equation}{Gl.}{Gl.}
    \Crefname{equation}{Gleichung}{Gleichungen}
    \crefname{table}{Tab.}{Tab.}
    \Crefname{table}{Tabelle}{Tabellen}
    \fi
  }
%    \end{macrocode}
% If option "web" is given to the class, do everything as described above. For
% option "print", hide links, i.e. make them black again.
%    \begin{macrocode}
  \ifthenelse{\boolean{ktx@web}}{%
    \relax % do nothing
  }{
    \hypersetup{hidelinks}
  }  
}{%
  \relax  % do nothing
}
%    \end{macrocode}
% \end{macro}
